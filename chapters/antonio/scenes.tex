\chapter{Scene Understanding}
\label{chapter:scene_understanding}

\section{Introduction}


% what might be cool is to have a running example of a photo of a scene and then show all the things you can infer about it, using off the shelf models, and then maybe talking about how each inference is useful in different ways, instances are good for tracking, affordances are good for robots, etc


Scene understanding refers to the analysis of the whole scene: the agents and objects in it, the relationships between them, the actions, the place, ...


\section{A Few Notes About Scene Understanding in Humans}

\subsection{Rapid Scene Recognition}

Molly Poter and her experiments on rapid scene recognition.

``Later she discovered that complex visual scenes can be perceived and understood much faster than anyone had previously recognized. She showed that subjects can identify the gist of a scene from an astonishingly brief presentation. Here Potter made innovative use of rapid serial visual presentation (RSVP).''

FIGURE: show one experiment

\subsection{Neural Processing}

The time it takes for us to recognize a scene is remarkable once we relate it to the structure of the neural pathways and the time it takes for a neuron to process its input. Simon Thorpe ...

Face area

Place fusiform area.

IT

\subsection{Remembering Images}

Another important aspect of human perception is memory.

A computer vision system can compare two images directly, even at the pixel level. A human can not do this because it can not look at two images simultaneously in parallel. A human needs to look at each image sequentially, and compare the two images from memory. Therefore, memory is a crucial aspect of human visual perception.

change blindness

The world as an outside memory.

\subsection{Recognition in Context}

Understanding a scene is an integrated process.

Some context illusions.

Out of context image.

Objects that belong to the same class have a common set of affordances, but the affordance of an object might change depending on context. As an example, if we use a card-box as a table, we might still call it {\em a box} but we understand the affordance of the box as being very different to its typical use.

Put the shoes in a place that is easy to find later.



\section{Scene Representations}

Schemas

Frames

Movies, stories.


Non-linguistic representations?

Scene graphs

Symbolic-representations


\section{Taskonomy}

Place classification

Recognizing locations

3D reasoning and decomposing the scene into separated objects. Understanding the rules of support, parts and attachments between objects.


\section{Concluding Remarks}

