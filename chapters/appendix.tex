

\appendix

\section{Matrix calculus}\label{appendix:matrix_calc:notational_conventions}
In this book, we adopt the following notational conventions. These conventions make the equations simpler, and that also means simpler implementations when it comes to actually writing these equations in code.

Vectors are represented as column vectors with shape $[N \times 1]$:
\begin{align}
    \mathbf{x} \triangleq
    \begin{bmatrix}
        x_1    \\
        x_2    \\
        \vdots \\
        x_N    \\
    \end{bmatrix}
\end{align}
If $y$ is a scalar and $\mathbf{x}$ is an $N$-dimensional vector, then the gradient $\frac{\partial y}{\partial \mathbf{x}}$ is a row vector of shape  $[1 \times N]$:
\begin{align}
    \frac{\partial y}{\partial \mathbf{x}} \triangleq
    \begin{bmatrix}
        \frac{\partial y}{\partial x_1} & \frac{\partial y}{\partial x_2} & \cdots & \frac{\partial y}{\partial x_N} \label{backprop:scalar_vector_deriv}
    \end{bmatrix}
\end{align}
If $\mathbf{y}$ is an $M$-dimensional vector and $\mathbf{x}$ is a $N$-dimensional vector then the gradient (also called the Jacobian in this case) is shaped as $[M \times N]$:
\begin{align}
    \frac{\partial \mathbf{y}}{\partial \mathbf{x}} \triangleq
    \begin{bmatrix}
        \frac{\partial y_1}{\partial x_1} & \frac{\partial y_1}{\partial x_2} & \cdots & \frac{\partial y_1}{\partial x_N} \\
        \vdots                            & \vdots                            & \vdots & \vdots                            \\
        \frac{\partial y_M}{\partial x_1} & \frac{\partial y_M}{\partial x_2} & \cdots & \frac{\partial y_M}{\partial x_N}
    \end{bmatrix}
\end{align}
Finally, if $\mathbf{W}$ is an $[N \times M]$ dimensional matrix, and $\mathcal{L}$ is a scalar, then the gradient $\frac{\partial \mathcal{L}}{\mathbf{W}}$ is represented as an $[M \times N]$ dimensional matrix (note that the dimensions are transposed from what you might have expected; this makes the math simpler later):
\begin{align}
    \frac{\partial \mathcal{L}}{\partial \mathbf{W}} & \triangleq
    \begin{bmatrix}
        \frac{\partial \mathcal{L}}{\partial \mathbf{W}_{11}} & \ldots & \frac{\partial \mathcal{L}}{\partial \mathbf{W}_{N1}} \\
        \vdots                                                & \ddots & \vdots                                                \\
        \frac{\partial \mathcal{L}}{\partial \mathbf{W}_{1M}} & \ldots & \frac{\partial \mathcal{L}}{\partial \mathbf{W}_{NM}} \\
    \end{bmatrix} \label{backprop:scalar_matrix_deriv}
\end{align}
Everything in this section is just definitions. There is no right are wrong to it. We could have used other conventions but we will see that these are useful ones.



